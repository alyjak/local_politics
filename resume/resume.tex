%%%%%%%%%%%%%%%%%%%%%%%%%%%%%%%%%%%%%%%%%
% Medium Length Professional CV
% LaTeX Template
% Version 2.0 (8/5/13)
%
% This template has been downloaded from:
% http://www.LaTeXTemplates.com
%
% Original author:
% Trey Hunner (http://www.treyhunner.com/)
%
% Important note:
% This template requires the resume.cls file to be in the same directory as the
% .tex file. The resume.cls file provides the resume style used for structuring the
% document.
%
%%%%%%%%%%%%%%%%%%%%%%%%%%%%%%%%%%%%%%%%%

%----------------------------------------------------------------------------------------
%   PACKAGES AND OTHER DOCUMENT CONFIGURATIONS
%----------------------------------------------------------------------------------------

\documentclass{resume} % Use the custom resume.cls style

\usepackage[left=0.4 in,top=0.3 in,right=0.4 in,bottom=0.3in]{geometry} % Document margins
\newcommand{\tab}[1]{\hspace{.2667\textwidth}\rlap{#1}}
\newcommand{\itab}[1]{\hspace{0em}\rlap{#1}}
\name{Andrew Lyjak} % Your name
\address{6205 N Oakland Ave, Indianapolis IN, 46220} % Your address
\address{+1 734 604 6163 \\ andrew.lyjak@gmail.com} % Your phone number and email

\begin{document}

%----------------------------------------------------------------------------------------
%   EDUCATION SECTION
%----------------------------------------------------------------------------------------

\begin{rSection}{Education}

{\bf ME Space Systems Engineering} \hfill {2009 - 2010}
\\
University of Michigan, GPA: 7.538/8.000

{\bf BE Aerospace Engineering} \hfill {2005 - 2009} \\
Minor in German \smallskip \\
\\
University of Michigan, GPA: 3.456/4.000
\end{rSection}

%----------------------------------------------------------------------------------------
%   TECHNICAL STRENGTHS SECTION
%----------------------------------------------------------------------------------------

\begin{rSection}{Interests}

  Design and Development Processes, Distributed Systems, Resilient Systems, Autonomy, Automation,\\
  Software Certification, Verification of Complex Systems, Model Driven Development, Continuous Integration,\\
  Agile Development

\end{rSection}

\begin{rSection}{Skills and Tools}
  Software Development Process Design, Software Process Audits, System
  Certification, Verification Planning, Fault Tolerance Analysis,
  Python, Javascript, HTML, CSS, Bash, Linux, R, C, C++, SQL, \LaTeX,
  ReStructuredText, Graph Analysis, git, Trac, Subversion, JIRA,
  Confluence
\begin{tabular}{ @{} >{\bfseries}l @{\hspace{6ex}} l }
\end{tabular}

\end{rSection}

%-------------------------------------------------------------------------------
%   PROJECTS

\begin{rSection}{Experience}

\begin{rSubsection}{Software Certification for NASA Safety and Quality Requirements} {2010 - Present}{}{}

\item Wrote the Flight Software Development Plan. This document defines SpaceX's
  custom agile development process for designing, developing, and verifying
  SpaceX flight software.
\item Designed and and wrote the Cargo Dragon Flight Software ``Computer Based
  Control System'' certification documentation. This material was used to
  demonstrate SpaceX compliance with NASA software safety requirements and also
  defined many of the software safety test and analysis activities executed for
  verification of software safety.
\item Perform process audits to verify software quality, and
  compliance with NASA requirements.
\item Maintain audits, and safety documents through all changes to
  system design and software.

\end{rSubsection}

\begin{rSubsection}{Independent Verification and Validation (IVV) Contract Management} {2010 - Present}{}{}

\item Serve as the technical point of contact for SpaceX's IVV Contracts for
  safety critical software.
\item Provide the contractor with sufficient data to effectively
  evaluate the safety of the flight software. Contractor evaluates
  system design, source code, and verification results.
\item Ensure the contractor's feedback is internally addressed and
  incorporated to create a better product.
\item manage contracts associated with IVV of flight software for Crew
  Dragon, Cargo Dragon, and SpaceX's Autonomous Flight Termination
  System.

\end{rSubsection}

%------------------------------------------------

\begin{rSubsection}{Software Process Tool Development}{2013 - 2014}{}{}

\item \textbf{branchdiff} - Developed an application to view the differences
  between two Subversion branches to facilitate merge decisions
  between them. Displays differences as commits or as the set tickets
  referenced within those commit messages.
\item Developed an application for viewing change over time for various Trac
  ticket queries.
\item Performed trade studies on various software development ticketing systems
  -- Trac, JIRA, Phabricator, Redmine. The study factored into
  SpaceX's decision to adopt JIRA across multiple business domains.
\item \textbf{ReadTheManual} - Installed, Modified, and Adminstered an internal
  fork of ReadTheDocs for use within the SpaceX Intranet. Used to
  distribute documentation for over 160 internal projects.
\item \textbf{Tracegraph} - Designed and Developed a protocol and
  library for defining systems relationships for verification tracking
  purposes. Library is used for compliance tracking of spacex
  processes to customer requirements.

\end{rSubsection}


%-------------------------------------------------


\begin{rSubsection}{Design and Development of the SpaceX Software
    Standard} {2013 - 2015}{}{}

\item Developed a software engineering standard for SpaceX. The
  standard provides a set of requirements that can be used for
  different classifications of software development, classifications
  include A, for safety and mission critical software, through D, used
  for desktop, R\&D, or other non-critical applications. The standard
  requirements are categorized to account for different aspects of
  software design and development, ranging from change management,
  risk management, verification processes, and technical and
  analytical requirements for safety critical software. The standard
  is used to evaluate all softare processes related to the NASA
  Commercial Crew contract.

\end{rSubsection}

%--------------------------------------------------

\begin{rSubsection}{Design, Development, and Execution of the SpaceX Fault\\
    Tolerance Analysis Process}{2016 - Present}{}

\item Co-developed the \textbf{Fault Tolerance Analyis Process}. This
  process is used to evaluate autonomous and/or operator controlled
  electromechanical systems. These analyses systematically assess a
  system's fault tolerance in order to determine the design's
  robustness as well as develop fault detection, isolation, and
  recovery algorithms for managing a fault tolerant system's redundant
  capabilities.
\item Developed test and analysis plans for verifying fault tolerance
  as defined through the Fault Tolerance Analysis products.
\item With a team of 6 other SpaceX Engineers, executed the Fault Tolerance
  Analysis Process against 25 separate autonomous control systems present
  within the Crew Dragon Architecture over the course of a year.

\end{rSubsection}


\end{rSubsection}

\end{rSection}
