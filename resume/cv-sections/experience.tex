% ----------------------------------------------------------------------------------------
%	SECTION TITLE
% ----------------------------------------------------------------------------------------

\cvsection{Experience}

% ----------------------------------------------------------------------------------------
%	SECTION CONTENT
% ----------------------------------------------------------------------------------------


\begin{cventries}

  % ------------------------------------------------

  \cventry
  {SpaceX}{Software Mission Assurance Engineer}{Hawthorne, CA}{2010 - Present}
  {}

  \cventry
  {Software Certification for NASA Safety and Quality Requirements}
  {}{}{2010 - Present}
  {
    \begin{cvitems}
    \item Wrote the Flight Software Development Plan. This document defines
      SpaceX's agile development process for designing, developing, and
      verifying flight software.
    \item Designed, wrote, and maintain the Cargo Dragon Flight Software
      ``Computer Based Control Systems'' documentation. This material is
      used to demonstrate compliance with NASA software safety
      requirements and also defines many of the software safety test and
      analysis activities executed for verification of software safety
      related to the Dragon cargo vehicle.
    \item Perform process audits to verify software quality, and
      compliance with NASA requirements for the Dragon cargo vehicle.
    \item Maintain audits, and safety documents through all changes to
      system design and software for each new commercial cargo resupply
      mission.
    \end{cvitems}
  }

  % ------------------------------------------------    

  \cventry
  {Independent Verification and Validation (IVV) Contract Management}{}{}{2010 - Present}
  {
    \begin{cvitems}
    \item Serve as the technical point of contact for SpaceX's IVV Contracts for
      safety critical software.
    \item Provide the contractor with sufficient data to effectively
      evaluate the safety of the flight software. Contractor evaluates
      system design, source code, and verification results.
    \item Ensure the contractor's feedback is internally addressed and
      incorporated to create a better product for each commercial crew and
      cargo mission.
    \item Manage contracts associated with IVV of flight software for Crew
      Dragon, Cargo Dragon, and the Autonomous Flight Termination
      System.
    \end{cvitems}
  }

  % ------------------------------------------------

  \cventry
  {Software Process Tool Development}{}{}{2012 - Present}
  {
    \begin{cvitems}
    \item \textbf{branchdiff} - Developed an application to view the differences
      between two Subversion branches to facilitate merge decisions
      between them. Displays differences as commits or as the set tickets
      referenced within those commit messages.
    \item Developed an application for viewing change over time for various Trac
      ticket queries.
    \item Performed trade studies on various software development
      ticketing systems. The study factored into SpaceX's decision to
      adopt JIRA across multiple business domains.
    \item \textbf{ReadTheManual} - Installed, modified, and adminstered an
      internal fork of ReadTheDocs for use within the SpaceX
      intranet. This service is used to build and distribute documentation
      for over 160 internal projects.
    \item \textbf{Tracegraph} - Designed and developed a library for
      defining systems relationships across information housed within
      multiple data silos. The library is used for verification tracking
      purposes to support compliance tracking of SpaceX processes to
      customer requirements.
    \end{cvitems}
  }

  % ------------------------------------------------

  \cventry
  {Design and Development of the SpaceX Software Standard}{}{}{2013 - Present}
  {
    \begin{cvitems}
    \item Developed a software engineering standard for SpaceX. The
      standard provides a set of requirements to be applied to different
      classifications of software development, classifications include A,
      for safety and mission critical software, through D, used for
      desktop, R\&D, or other non-critical applications. The standard is
      used to evaluate the quality of all software processes related to the
      Commercial Crew system.
    \end{cvitems}
  }

  % ------------------------------------------------

  \cventry
  {Fault Tolerance Analysis}{}{}{2016 - Present}
  {
    \begin{cvitems}
    \item Co-developed the \textbf{Fault Tolerance Analyis Process}. This
      process is used to evaluate autonomous and/or operator controlled
      electromechanical systems. Analysis is used to assess the fault
      tolerance of a design in order to determine its capabilities and
      develop fault detection, isolation, and recovery logic for managing
      redundant capabilities.
    \item Developed test and analysis plans for verifying fault tolerance
      as defined through the Fault Tolerance Analysis products.
    \item With a team of 6 other engineers, executed the Fault Tolerance
      Analysis Process against 25 separate autonomous control systems
      within the Crew Dragon Architecture over the course of a year.
    \end{cvitems}
  }
  % ------------------------------------------------
\end{cventries}
