\documentclass[a4paper]{article}
% generated by Docutils <http://docutils.sourceforge.net/>
\usepackage{cmap} % fix search and cut-and-paste in Acrobat
\usepackage{ifthen}
\usepackage[T1]{fontenc}
\usepackage[utf8]{inputenc}

%%% Custom LaTeX preamble
% PDF Standard Fonts
\usepackage{mathptmx} % Times
\usepackage[scaled=.90]{helvet}
\usepackage{courier}

%%% User specified packages and stylesheets

%%% Fallback definitions for Docutils-specific commands

% rubric (informal heading)
\providecommand*{\DUrubric}[1]{%
  \subsubsection*{\centering\textit{\textmd{#1}}}}

% hyperlinks:
\ifthenelse{\isundefined{\hypersetup}}{
  \usepackage[colorlinks=true,linkcolor=blue,urlcolor=blue]{hyperref}
  \usepackage{bookmark}
  \urlstyle{same} % normal text font (alternatives: tt, rm, sf)
}{}
\hypersetup{
  pdftitle={Bihome Alpha Design Brief},
}

%%% Title Data
\title{\phantomsection%
  Bihome Alpha Design Brief%
  \label{bihome-alpha-design-brief}}
\author{}
\date{}

%%% Body
\begin{document}
\maketitle

\DUrubric{PINA NatureWise Design Contest}

Climate change, coupled with ever growing global supply chains are continually increasing the strain
on the ecological webs that sustain life on planet Earth. Traditionally, suburban living has been
seen as part of the problem. Our entry, named BiHome Alpha, attempts to change that idea by
exploring what sort of sustainable living practices may be achievable within an ordinary suburban
landscape without drastically changing the amount of time or money a typical homeowner puts into
their property. In this design, we apply permaculture principles to a 0.6 acre partially wooded
property. Four focus areas provide the theme of the plan:

\begin{itemize}
\item Collection of performance data in order to compare inputs and outputs between the baseline and
transformed landscape. Input data will include time, money, and external material for initial
development and maintenance of both. Output data will include metrics of human and ecological
interest, including analysis of soil health and carbon content, water management, as well as
measuring the yield of the human-consumable goods produced within the landscape.

\item Intentional integration of mycelium into newly developed plots to aid in plant growth, increase
soil health, and provide edible mushrooms.

\item Integration of cost effective water management and retention techniques, and

\item Carbon sequestration through growth of the soil and trees within the property.
\end{itemize}

Other than these themes, the design incorporates an indoor set up for growing seedlings, a small
aquaponics tank, a spiral herb garden, multiple raised beds planted with annual guilds, a small food
forest of apple and cherry, and finally a copse of fast growing deciduous trees that can be coppiced
for lumber. Included in this lumber forest are American Chestnut saplings breeded by the American
Chestnut Foundation, and Paulownia, which is often referred to as the fastest growing deciduous tree
in the world.

BiHome Alpha serves as the primary testbed for the web application we are developing, named
bihome. This application enables users to transform their yard from a sterile grass lawn into a
usable, practical permaculture based development -{}- all for a similar amount of time and effort it
has taken a traditional homeowner to maintain a lawn-focused landscape. The app shows homeowners
rational reasons why the current practice of suburban grass development makes no economic sense, and
it cuts down the amount of research and planning necessary to execute permaculture principles. Of
course when the users are ready to learn more, the app provides easy access to all existing
communities that will gladly teach a curious new member.

We have attempted to use this contest entry to showcase the asthetic of the app, and will be using
the data collected through execution of the plan to further improve the app's functionality before
releasing it to a wider audience.

\end{document}
